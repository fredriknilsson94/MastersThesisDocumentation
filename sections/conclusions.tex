\chapter{Conclusions and Future Work}
\label{cha:conclusions}

This thesis has shown that it is possible to estimate the heading (orientation and angular rate) of a vehicle using a mono camera system.
There is still some continued work that has to be done in order to achieve the same level of performance as a stereo camera system.

\section{Conclusions}
Through this thesis, a proposed model has been presented in order to solve the heading estimation problem.
Given the results presented in this thesis and the related research described in \Sectionref{sec:relatedresearch}, the thesis has shown that it is possible to estimate the heading of a vehicle using a mono camera system.
This thesis has shown, through Monte Carlo simulations, that given the proper measurements and measurements with plausible noise variance level, the proposed method is able to estimate the orientation and angular rate of a target vehicle.
The modelling strategy, by modelling the target as a rectangle, seems like a suitable model for describing the target.
It captures the shape of the target in a good way and the rotation is easily defined in the model.
The model is also suitable for measurements of the target vehicle's corners.
Although no other models have been tested in this thesis, the rectangle model seems like an eminent model choice.

The results with real-world data were not as satisfying as the best simulated results.
Due to the high variance of the constructed angular rate measurements, the same gratifying results achieved through simulations could not be obtained with real-world data.
By adding the target vehicle's corners as measurements, the state estimates could further be improved in the case of a turning target.

\newpage

By recalling the objective of this thesis including the questions that were stated in \Chapterref{cha:intro}, answers can now be provided in order to finalize the thesis.

\begin{itemize}
    \item Can the heading of a vehicle be estimated using a mono camera system?

    Yes, the heading of a vehicle can be estimated using a mono camera system if the proper measurements are available with high enough reliability.

    \item What kind of algorithms and models are suitable for solving the problem?

    By modelling the target vehicle as a rectangle and using the proposed measurements, a suitable algorithm has been constructed in order to solve the heading estimation problem.

    \item How well does the estimates from a mono camera system perform compared to those from a stereo camera system?

    From the results in this thesis, the mono camera system can not quite catch up with the performance of a stereo camera system.
    The result could be improved by
	\begin{enumerate}[label=\roman*)]
		\item increasing the number of feature points or improve the feature point correspondence between two frames,
		\item further investigate the homography estimation method,
		\item constructing a method which can generate measurements of the corners.
	\end{enumerate}
	If these improvements are implemented, then the mono camera system can have a reasonable chance of competing in performance with a stereo camera system.
\end{itemize}

\section{Suggestions for Future Work}
There are several things which could be improved in the future.
Right now, the mono camera system is still lacking behind in the capability of estimating the heading of vehicles, compared to a stereo camera system.
By looking at the thesis limitations, several tasks for future work can be stated.
However, a couple of more interesting future tasks are mentioned here.

\subsection{Homography Estimation}
By further investigating the homography estimation method, perhaps the noise level could be reduced.
Data normalization, \eg Hartley normalization \cite{Nordberg:2015}, is one concept which might improve the homography estimation.
Another method is to reduce the number of parameters to estimate in the homography matrix.
Since the homography describes the rotation in three dimensions it should be possible to reduce the number of free parameters.
A vehicle can (in normal cases) only rotate in one of the three dimensions.

\subsection{Feature Point Correspondence}
Testing another method for the feature point detection, description, matching and tracking might also be a strategy to obtain better homography estimations.
Since the quality of the estimated yaw rate from the homograpy did not match the quality of the estimated yaw rate in \cite{Gabb:2013}, improving the KLT feature point tracker could be one way forward.
The feature point detection could also be analysed further.
As discussed in \Sectionref{sec:homographyestimationresults}, the quality and spatial location of the detected feature points have an important role for the homography estimation.
One thing to have in mind is that in this thesis the feature point tracker where implemented in \matlab.
If it should run in a mono camera system, different aspects of the computation complexity might have to be taken into consideration.

\subsection{Corner Measurements}
This thesis has shown that corner measurements could improve the heading estimation in a mono camera system.
Coming up with a method to construct a detector, which could detect the corners in the image, could therefore be a highly interesting part of the future work.