Advanced driver assistance systems (\abbrADAS) are a popular and evolving area of research and development.
By providing assistance to the vehicle drivers, \abbrADAS could significantly reduce the number of traffic accidents since 90 \% of all accidents are caused by the human factor.
\abbrADAS with cameras provides a wide field and view and thanks to today's advanced image processing techniques, lots of information can be extracted from the camera image.
This thesis has come up with a method of estimating the heading of vehicles using a mono camera system.
The proposed method consists of an Extended Kalman filter with a constant velocity motion model to predict the vehicle's path.
By inputting classification measurements from machine learning algorithms together with angular rate measurements into the filter, Monte Carlo simulations performed in this thesis have shown promising results.
The results on real-world data indicates that the method on how to construct the angular rate measurements must be improved in order to reach the same results as obtained from the simulations.
An additional measurement type of the vehicle's corners where introduced in order to further provide the filter with information.
The thesis has shown that the mono camera system needs further improvements in order to compete with the same level of performance compared to a stereo camera system.