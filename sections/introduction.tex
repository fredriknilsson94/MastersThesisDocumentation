\chapter{Introduction}
\label{cha:intro}
This \ms deals with how to estimate the heading (orientation and angular rate) of other vehicles, using a camera mounted in the ego vehicle.
The problem is easily solved by using a camera system consisting of two side-by-side mounted cameras.
A camera system consisting of a single camera is cheaper than a system of two cameras which is a great motivation to why this problem is of interest to investigate.

\section{Background}
Today, autonomous driving and driver assistance systems is a popular area of research and development.
Especially, since it has the ability to make driving more safe. 
One organization, working with evaluation of car safety, is Euro NCAP.
They have created a five-star safety rating system, in order to guide and assist the customers when they are purchasing a new car.
The rating is based on a series of vehicle tests, representing everyday traffic scenarios. 
In order for the car manufactures to get a good rating by Euro NCAP, they need intelligent safety systems in their cars.
This requires safety systems capable of handling \eg interurban, city and pedestrian automatic emergency braking (\abbrAEB), \ie a system that is capable of mitigating or avoiding collisions with pedestrians and cars.
In the future, the automatic emergency steering (\abbrAES) and \abbrAEB systems will have more demanding requirements in order for the car manufactures to reach the highest safety ratings \cite{EuroNCAP:2017}.

One possible technical solution to reach the safety requirements is to use vision based safety assistance systems.
By using a camera system, capable of capturing a wide scene in front of the vehicle combined with image processing techniques, good knowledge about the surrounding environment of the vehicle can be obtained.
Vision safety systems are typically constructed with either one or two cameras, denoted mono camera systems and stereo camera systems, respectively.
One of the great advantages from using a stereo camera system, instead of a mono camera system, is the ability to gain depth information from the disparity mapping \citep{Sivaraman:2013}.
This is exactly the same ability as the human eyes have when we observe the world, \ie reconstruction of \spacedim{3} images.
This significantly improves the ability to get information about how vehicles are \eg oriented.
One benefit of knowing how vehicles are oriented, and how they are rotating, is that more information can be used when predicting the future path of the vehicles.
Especially, since \eg a vehicle has constraints that limits its movements, knowing the orientation and angular rate can give large benefits.

The lack of distance information is a challenge that must be handled when developing a mono camera system.
It is however not impossible to gain some depth information from mono camera systems.
One method is to assume a width or height of an observed object in the image.
Another method is to use machine learning algorithms to recover the depth information from images \cite{Saxena:2008}.
Today, advanced driver assistance systems (\abbrADAS) can fuse the information from a mono camera system together with data from \eg a radar system.
This utilizes the cameras capability of detecting objects in a wide field of view together with the distance information from the radar.

Stereo cameras seem like the obvious choice.
However, they are more expensive since more hardware is used.
For a car that is already expensive, this might not be a problem.
Although, in order to improve the overall traffic situation more cars must use a driving assistance system.
The fact is that more than 90 \% of all accidents on the road are caused by the human error \cite{EuroNCAP:2017}!
The cheaper mono camera system is a competitive alternative to use in order to equip more cars with \abbrADAS.
In order for the mono camera system to be a reasonable substitute for the stereo camera system, functionality that exists in a stereo camera system must also be a feature in the mono camera system.
Thus, in order to make vision systems more available for all car manufactures and all car models, further development of the mono camera systems is necessary.

\section{Purpose and Objective}
The purpose of this \ms is to investigate if and how the heading of vehicles can be estimated in a mono camera system.
Further, the thesis shall analyse how well the mono camera system performs compared to a stereo camera system.
From this, the thesis can preferably come up with some conclusions about the possibilities of replacing a stereo camera system with a mono camera system.

The main idea is to research the subject and find promising algorithms for heading estimation, implement and evaluate one algorithm and compare the results with the results from a stereo camera system.
The objective can be formulated into several questions:

\begin{itemize}
	\item Can the heading of a vehicle be estimated using a mono camera system?

	\item What kind of algorithms and models are suitable for solving the problem?

	\item How well do the estimates from a mono camera system perform compared to those from a stereo camera system?
\end{itemize}

\section{Limitations}
In order to make the thesis feasible, some limitations must be accepted. They are:

\begin{itemize}	
	\item The heading estimation algorithm will only be evaluated on cars.

	If the rotation is successfully estimated on a car, generalizations to other vehicles, \eg buses and trucks should not be that complicated.

	\item The size of the tracked car is assumed to be known.

	For simplification and in order to have some  distance information, it is assumed that we have knowledge about the width and length of the tracked car.
	It is future work to investigate how incorrect assumptions about the size affects the final results.

	\item The algorithm is not required to work in real-time.

	Since it is initially unknown if it is possible to estimate the heading of vehicles in a mono camera system, the focus is on proof-of-concept and not on real-time performance.
	However, the real-time aspect should be kept in mind when designing the algorithm.

    \item The tracking algorithm does not has to be able to complete all necessary steps automatically.

    The algorithm can \eg be informed if the host car is observing the front or rear of the target car.

	\item It is assumed that we have perfect knowledge about the ego car's ego motion.

	By assuming that the knowledge about the ego car's ego motion is perfect, the step of compensating the states of the target, \ie the position and orientation, is (almost) trivial.
	Therefore, this thesis has dealt only with tracking performed from a car which is standing still, in order to simulate perfect ego motion.
\end{itemize}

\newpage

\section{About Veoneer}
Veoneer is a worldwide leader company in automotive safety.
During 2018, the company Autoliv split up the two business areas, passive and active safety, into two separate companies.
Autoliv continued to be responsible for the passive business area while Veoneer took over the responsibility for the active safety.
Veoneer constructs software for \abbrADAS, night-vision systems, radar and LiDAR systems as well as hardware constructions.
\cite{Veoneer:2018}

\section{Thesis Outline}
The thesis is structured the following way:
\begin{description}
    \item[\Chapterref{cha:theory}] includes relevant theory about different algorithms that can possibly be used to solve the problem and other theory necessary to understand the thesis.

    \item[\Chapterref{cha:method}] describes the proposed method regarding target modelling and tracking algorithm for solving the problem of heading estimation.

    \item[\Chapterref{cha:result}] presents the result of the algorithm evaluation and comparison with a stereo camera system.

    \item[\Chapterref{cha:conclusions}] concludes the thesis and describes some suggestions for future work.
\end{description}
